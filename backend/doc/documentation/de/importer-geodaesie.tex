\section{Geodatenimport}

Der Geodaten Importer ist ein in der Programmiersprache Python
geschriebenes Kommandozeilen Werkzeug zum Import von Shapefiles in
eine Datenbank.
Zum Lesen der Shapefiles und zum Schreiben der Geodaten
in die Datenbank wird die GDAL-Bibliothek verwendet.
Um Daten in eine Oracle Datenbank zu importieren ist es nötig, dass
GDAL und GDAL-Python-Bindings mit Oracle-Unterstützung installiert
sind. Bei der Verwendung von PostgreSQL entfällt dieser Schritt.
Weitere Details hierzu befinden sich im
Kapitel \ref{Systemanforderungen} und \ref{Installationsanleitung}.

Der Importer kann mit einem Shellscript von der Kommandozeile gestartet werden
(siehe Kapitel \ref{Starten des Geodaten Importers}). Nach dem Start wird anhand der
Konfiguration festgestellt, welche Klassen von Shapefiles aus dem Dateisystem
importiert werden sollen. Für jede Klasse gibt es einen speziellen
Parser, der die speziellen Attribute eines Shapefiles liest und in die entsprechende
Relation der Datenbank schreibt. Die Parser sind speziell auf das
Dateisystem der BfG ausgerichtet. So wird beispielsweise erwartet, dass die Shapefiles der
Gewässerachse im Ordner $Geodaesie/Flussachse+km$ liegen. Weitere Informationen zu
den einzelnen Parsern sind Kapitel \ref{Beschreibung der Parser} zu
entnehmen.

Damit die Geodaten eines Shapes später eindeutig in der Datenbank identifiziert
werden können, wird für jede Geometrie der Pfad des Shapes im Dateisystem
im Datenbankfeld 'path' gespeichert. Anwendungen, die auf der Datenbank
aufbauen, können die Geodaten eines Shapefiles später anhand dieses Merkmals
gruppieren und anzeigen.

Bitte beachten Sie, dass der Geodaten Importer aufgrund der eingesetzten
Technologien derzeit nicht in der Lage ist, lesend auf die Oracle-Datenbank
zuzugreifen. Entsprechend kann beim Import nicht festgestellt werden, ob sich
die Daten eines Shapefiles bereits in der Datenbank befinden, oder nicht.
Ein erneuter Import der Geodaten würde also dazu führen, dass Geometrien doppelt in der
Datenbank abgelegt werden.

\subsection{Koordination-Transformation}
Für die Transformation der Daten verwendet GDAL wiederum die PROJ4-Bibliothek.
Die Daten werden vor dem Schreiben in die Datenbank alle
in die Gauß-Krüger-Projektion Zone 3 (EPSG-Code 31467) transformiert.
Ist für die zu importierenden Daten keine Projektion ersichtlich
(fehlende \textit{*.prj}-Datei), so findet keine Transformation statt.
Dies führt nur zu Problemen mit dem Fachdienst FLYS, falls die Daten nicht
bereits in der genannten Projektion vorlagen.

Im Falle der Digitalen Geländemodelle (DGM) findet keine Transformation statt,
da zu diesen lediglich Metadaten in der Datenbank gespeichert werden
(siehe Kapitel \ref{dgm_parser}),
während die Daten selbst von der Anwendung Dive4Elements River
aus dem Dateisystem geholt werden.
Für Berechnungen mit den DGM werden die Geometrien aus der Datenbank
in Dive4Elements River in die Projektion des jeweiligen DGM transformiert.
Daher ist es besonders wichtig, dass die Angaben des EPSG-Codes
in der Spalte SRID in DGMs.csv korrekt sind (siehe Kapitel \ref{dgm_parser})

\subsection{Logfile}
Der Erfolg oder Misserfolg eines Shape-Imports wird
im Logfile vermerkt. Folgende Einträge können dem Logfile
entnommen werden:

\textbf{INFO: Inserted \# features}
\\Gibt die Anzahl der erfolgreich importierten Features an.

\textbf{INFO: Failed to create \# features}
\\Gibt die Anzahl der Features an, die nicht importiert werden konnten.

\textbf{INFO: Found 3 unsupported features of type: '...'}
\\Gibt die Anzahl der Features an, die aufgrund ihres Datentyps nicht importiert
werden konnten. Wenn etwa Punkte erwartet wurden aber sich im Shapefile
Polygone befanden.

\textbf{INFO: Did not import values from fields: '...' ...}
\\Der Importer schreibt neben der geographischen Information weitere
Attribut-Daten in die Datenbank.
Attribut-Spalten die nicht importiert wurden (z.B. auf Grund
von Tippfehlern oder unterschiedlicher Schreibweise),
werden wie angegeben im Logfile aufgeführt.

\textbf{ERROR: No source SRS given! No transformation possible!}
\\Das Shapefile enthält keine Information, in welcher Projektion die Geometrien
vorliegen. Es findet keine Transformation in die Zielprojektion statt. Bitte
beachten Sie, dass FLYS diese Geometrien später ggf.\ nicht korrekt darstellen
kann.

\textbf{ERROR: Unable to insert feature: DETAIL}
\\Beim Lesen eines Features ist ein Fehler aufgetreten.
Das Feature konnte nicht in die Datenbank geschrieben werden.

\textbf{ERROR: Exception while committing transaction}
\\Beim Abschluss des Schreib-Vorgangs in die Datenbank ist ein unerwarteter
Fehler aufgetreten. Die Features des Shapes sind nicht importiert worden.

\textbf{ERROR 1: ORA-01017: invalid username/password; logon denied}
\\Es konnte keine Verbindung zur Oracle Datenbank hergestellt werden. Prüfen Sie
die Verbindungseinstellungen.

Weitere Fehler, die von der Oracle-Datenbank kommen, können ebenfalls im
Logfile angezeigt werden.


\subsection{Beschreibung der Parser}
\label{Beschreibung der Parser}

Wie im letzten Kapitel beschrieben, sind die Parser speziell an das Dateisystem
der BfG angepasst. Im Folgenden werden zu jedem Parser folgende Informationen
angegeben:

\textbf{Pfad}
\\Der Pfad, in dem die Shapefiles im Dateisystem abgelegt sein müssen (ausgehend
vom Gewässer Verzeichnis).

\textbf{Geometrie}
\\Der Geometrie Typ, der für diese Klasse von Shapefiles erwartet wird.

\textbf{Attribute}
\\Eine Liste der Attribute, die vom Parser aus dem Shape gelesen werden.
In Klammern als alternativ bezeichnete Attribut-Namen werden in
das gleiche Datenbankfeld geschrieben, wie das vorgenannte Feld.
Die alternativen Namen werden vom Importer zusätzlich unterstützt,
um Dateien aus dem heterogenen Bestand der BfG unverändert
importieren zu können.

Zudem werden Datenbank-Attribute beschrieben, die nicht direkt aus
Attribut-Spalten des Shapefiles gelesen werden.

\subsubsection{Achsen}
\hspace{5mm}
\begin{tabular*}{155mm}[t]{l@{\extracolsep\fill}p{125mm}}
Pfad        &   Geodaesie/Flussachse+km \\
Geometrie   &   LINESTRING, MULTILINESTRING \\
\end{tabular*}

Das Attribut 'name' wird auf den Namen des Shapefiles gesetzt.
Zusätzlich wird das Attribut 'kind\_id' gesetzt, welches
für die aktuelle Achse (\textit{achse.shp}) 1 ist
und für sonstige Achsen (weitere Linien-Shapes) 2.

Hinweis:
Da die Layer-Konfiguration (Mapfiles) für die Flussachsen beim Start des
Artefakt-Servers generiert werden, muss der Artefakt-Server neugestartet
werden um die Flussachse in der Anwendung darstellen zu können.

\subsubsection{Hydr. Grenzen}
\hspace{5mm}
\begin{tabular*}{155mm}[t]{l@{\extracolsep\fill}p{125mm}}
Pfad        &   Hydrologie/Hydr.Grenzen \\
Geometrie   &   LINESTRING, MULTILINESTRING, POLYGON, MULTIPOLYGON \\
Attribute   &   SECTIE, STROVOER \\
\end{tabular*}

Das Attribut 'name' wird auf den Namen des Shapefiles gesetzt.
Das Attribut 'kind' wird 1 gesetzt für Daten aus dem
Unterverzeichnis \textit{Linien/BfG},
2 für Daten aus \textit{Linien/Land},
3 für Daten aus \textit{Sonstige}
und für alle übrigen 0.
Ausgenommen sind Dateien, in deren Namen 'Talaue'
(Groß-Klein-Schreibung irrelevant) vorkommt.

Linien und Polygone werden in der Datenbank in unterschiedlichen
Tabellen gespeichert.

\subsubsection{Bauwerke}
\hspace{5mm}
\begin{tabular*}{155mm}[t]{l@{\extracolsep\fill}p{125mm}}
Pfad        &   Geodaesie/Bauwerke \\
Geometrie   &   LINESTRING \\
Attribute   &   Name (alternativ: KWNAAM),
  km (alternativ: station, wsv-km),
  z (alternativ: Höhe, Hoehe, m+NHN)\\
\end{tabular*}

Das Attribut 'name' wird auf den Namen des Shapefiles gesetzt.
Das Attribut 'kind\_id' ist 0 für Sonstige,
1 für Brücken, 2 für Wehre, 3 für Pegel.
Es wird aus dem Dateinamen hergeleitet
(\textit{bruecken.shp, wehre.shp, pegel.shp},
teilweise auch alternative Schreibweisen unterstützt)
oder je Feature gesetzt, wenn in einer Attributspalte
die Werte 'bruecke' und 'wehr'
(teilweise auch alternative Schreibweisen unterstützt) vorkommen.
Ausgenommen sind Dateien, in deren Namen 'Buhnen'
(Groß-Klein-Schreibung irrelevant) vorkommt.

\subsubsection{Querprofilspuren}
\hspace{5mm}
\begin{tabular*}{155mm}[t]{l@{\extracolsep\fill}p{125mm}}
Pfad        &   Geodaesie/Querprofile \\
Geometrie   &   LINESTRING \\
Attribute   &   KILOMETER (alternativ: KM, STATION), ELEVATION \\
\end{tabular*}

Das Attribut 'kind\_id' wird 1 gesetzt für die Datei \textit{qps.shp} (aktuelle Querprofilspuren)
und 0 für alle weiteren.

\subsubsection{Festpunkte}
\hspace{5mm}
\begin{tabular*}{155mm}[t]{l@{\extracolsep\fill}p{125mm}}
Pfad        &   Geodaesie/Festpunkte \\
Geometrie   &   POINT \\
Attribute   &   KM (alternativ: ELBE\_KM), X, Y, HPGP (alternativ: ART) \\
\end{tabular*}

Das Attribut 'name' wird auf den Namen des Shapefiles gesetzt.

\subsubsection{Hochwassermarken}
\hspace{5mm}
\begin{tabular*}{155mm}[t]{l@{\extracolsep\fill}p{125mm}}
Pfad        &   Hydrologie/HW-Marken/hw-marken.shp \\
Geometrie   &   POINT \\
Attribute   &   Ort (alternativ: Pegel),
  km (alternativ: station, wsv-km, FlussKm),
  z (alternativ: z mit anschließender Zahl, m+NHN)\\
\end{tabular*}

Groß-Klein-Schreibung im Dateinamen ist irrelevant.
Für das Attribut 'year' wird im Dateinamen nach einer Jahreszahl
nach folgendem Muster gesucht: \textit{\_YYYY\_} oder \textit{-YYYY-}.
Gelingt dies nicht, erscheint im Logfile die Warnung
'Could not extract year from filename: ...'.

\subsubsection{Talaue}
\hspace{5mm}
\begin{tabular*}{155mm}[t]{l@{\extracolsep\fill}p{125mm}}
Pfad        &   Hydrologie/Hydr.Grenzen \\
Geometrie   &   POLYGON, MULTIPOLYGON \\
\end{tabular*}

Es werden nur Dateien betrachtet, in deren Namen das Wort 'Talaue'
(Groß-Klein-Schreibung irrelevant) vorkommt.
Das Attribut 'name' wird auf den Namen des Shapefiles gesetzt.
Das Attribut 'kind\_id' wird 1 gesetzt für die Datei \textit{talaue.shp} (aktuelle Talaue)
und 0 für alle weiteren.

\subsubsection{Hochwasserschutzanlagen}
\hspace{5mm}
\begin{tabular*}{155mm}[t]{l@{\extracolsep\fill}p{125mm}}
Pfad        &   Hydrologie/HW-Schutzanlagen \\
Geometrie   &   LINESTRING, MULTILINESTRING, POINT \\
Attribute   &   Name, Art, Quelle, Anmerkung, Stand, Verband,
                km (alternativ: Deich\_km), Bereich,
                Hoehe\_ist, Hoehe\_soll, WSP\_Bfg100, Bundesland
                (Teilweise auch alternative Schreibweisen unterstützt)\\
\end{tabular*}

Das Attribut 'name' wird auf den Namen des Shapefiles gesetzt,
wenn kein Attribut 'Name' im Shapefile vorhanden ist.
Das Feld 'kind\_id' wird per Default auf 2 (für Damm) gesetzt.
Wird ein Attribut 'ART' im Shapefile gefunden,
so wird 'kind\_id' entsprechend dieses Feldes gesetzt
(1 für die Werte 'Durchlass', 'Rohr1', 'Rohr 1', 'Rohr 2',
2 für die Werte 'Damm', 'Deich', 'Hochufer', 'Hauptdeich', 'Sommerdeich',
3 für den Wert 'Graben').
Es wird versucht das Bundesland aus dem Dateinamen zu ermitteln,
wenn das Shapefile kein Attribut 'Bundesland' enthält.

Linien und Punkte werden in der Datenbank in unterschiedlichen
Tabellen gespeichert.

\subsubsection{Buhnen}
\hspace{5mm}
\begin{tabular*}{155mm}[t]{l@{\extracolsep\fill}p{125mm}}
Pfad        &   Geodaesie/Bauwerke/Buhnen.shp \\
Geometrie   &   POINT \\
Attribute   &   station (alternativ: km, wsv-km),
                z (alternativ: Hoehe, Höhe, m+NHN) \\
\end{tabular*}

Das Attribut 'kind\_id' wird für
Buhnenkopf (\textit{bkl, bkr, bk}) 0,
für Buhnenfuß (\textit{bfl, bfr, bf}) 1 und
für Buhnenwurzel (\textit{bwl, bwr, bw}) 2 gesetzt,

\subsubsection{Stationierung}
\hspace{5mm}
\begin{tabular*}{155mm}[t]{l@{\extracolsep\fill}p{125mm}}
Pfad        &   Geodaesie/Flussachse+km/km.shp \\
Geometrie   &   POINT \\
Attribute   &   km (alternativ: KM), landkm \\
\end{tabular*}

Das Attribut 'name' wird auf den Namen des Shapefiles gesetzt.

\subsubsection{Überschwemmungsfläche}
\hspace{5mm}
\begin{tabular*}{155mm}[t]{l@{\extracolsep\fill}p{125mm}}
Pfad        &   Hydrologie/UeSG \\
Geometrie   &   POLYGON, MULTIPOLYGON \\
Attribut    &   diff, count, area, perimeter, GEWAESSER \\
\end{tabular*}

Das Attribut 'name' wird auf den Namen des Shapefiles gesetzt.
Das Attribut 'kind' wird nach folgendem Schema gesetzt:

\hspace{5mm}
\begin{tabular}[t]{ll}
Unterverzeichnis & Wert \\
\textit{Berechnungen/Aktuell/BfG} & 111 \\
\textit{Berechnungen/Aktuell/Bundesländer} & 112 \\
\textit{Berechnungen/Potentiell/BfG} & 121 \\
\textit{Berechnungen/Potentiell/Bundesländer} & 122 \\
\textit{Messungen} & 200 \\
\end{tabular}

Das Attribut 'source' wird auf den Namen des Verzeichnisses gesetzt,
indem sich das jeweilige Shapefile befindet.

\subsubsection{Metadaten zu Digitalen Gelände-Modellen}
\label{dgm_parser}
\hspace{5mm}
\begin{tabular*}{155mm}[t]{l@{\extracolsep\fill}p{125mm}}
Pfad        &   ../DGMs.csv \\
Attribut    &   Projektion, Höhenstatus, Format, Bruchkanten,
                Auflösung, SRID, Pfad\_Bestand,
                km\_von, km\_bis, Jahr\_von, Jahr\_bis \\
\end{tabular*}

Aus der Spalte 'Gewässer' in DGMs.csv wird entnommen,
für welches Gewässer das angegebene DGM verwendet wird.
Die Spalte muss daher den exakt gleichen Namen enthalten
wie in der *.gew-Datei des Gewässers angegeben
(siehe auch Kapitel \ref{start-hydr}).
Die eigentlichen Geo-Daten der DGM werden nicht in die Datenbank importiert.
Diese werden von der Anwendung Dive4Elements River aus dem Dateisystem geholt.

\subsection{Konfiguration}
\label{Konfiguration}
Der Geodaten Importer kann über das Skript \textit{./run\_geo.sh}
konfiguriert werden. Öffnen Sie die Datei mit einem Texteditor Ihrer Wahl
und passen Sie ggf.\ folgende Variablen an:

\textbf{HOST}
\\Der Host der Datenbank.

\textbf{BACKEND\_NAME}
\\Der Name der Datenbank Instanz.
Beispielsweise \textit{XE} bei einer Oracle XE Instanz.

\textbf{USER}
\\Der Nutzer, der zum Verbinden zur Datenbank verwendet wird.

\textbf{PASS}
\\Das Passwort für USER zum Verbinden zur Datenbank.

In den weiteren Zeilen werden weitere Optionen definiert, die bei Bedarf angepasst
werden können. Falls nicht anders angegeben, können die Optionen mit den Werten
`0` und `1` belegt werden.

\textbf{VERBOSE}
\\Dieser Wert gibt die Granularität der Log-Ausgaben während des
Imports an. Je höher der Wert, desto mehr Informationen werden
in das Logfile geschrieben. Aktuell sind die Werte `0`, `1` und
`2` definiert. Wird der Wert `0` gesetzt, werden nur Fehler und
Warnungen in das Logfile geschrieben. Bei `1` werden neben
Fehlern und Warnungen auch Infos in das Logfile geschrieben. Bei
`2` werden sämtliche Ausgaben des Programms geschrieben. Dieser
Modus ist hauptsächlich für die Entwicklung gedacht.

\textbf{OGR\_CONNECTION}
\\Hiermit kann direkt ein beliebiger Verbindungs-String angegegeben
werden, welcher dann anstatt HOST, USER und PASS verwendet wird.
Diese Option wird direkt an die OGR-Bibliothek weitergegeben und ermöglicht
verbesserte Tests und Entwicklung mit verschiedenen Backends.

\textbf{SKIP\_AXIS}
\\Bei gesetztem Wert `1` werden keine Flussachsen importiert.

\textbf{SKIP\_KMS}
\\Bei gesetztem Wert `1` werden keine Stationierungen importiert.

\textbf{SKIP\_CROSSSECTIONS}
\\Bei gesetztem Wert `1` werden keine Querprofilespuren importiert.

\textbf{SKIP\_FIXPOINTS}
\\Bei gesetztem Wert `1` werden keine Festpunkte importiert.

\textbf{SKIP\_BUILDINGS}
\\Bei gesetztem Wert `1` werden keine Bauwerke importiert.

\textbf{SKIP\_FLOODPLAINS}
\\Bei gesetztem Wert `1` werden keine Talauen importiert.

\textbf{SKIP\_HYDR\_BOUNDARIES}
\\Bei gesetztem Wert `1` werden keine hydrologischen Grenzen importiert.

\textbf{SKIP\_HWS\_LINES}
\\Bei gesetztem Wert `1` werden kein Hochwasserschutzanlagen  (Liniendaten) importiert.

\textbf{SKIP\_HWS\_POINTS}
\\Bei gesetztem Wert `1` werden kein Hochwasserschutzanlagen (Punktdaten) importiert.

\textbf{SKIP\_UESG}
\\Bei gesetztem Wert `1` werden keine Überschwemmungsflächen importiert.

\textbf{SKIP\_DGM}
\\Bei gesetztem Wert `1` werden keine Metadaten zu Digitalen Geländemodellen importiert.

\textbf{SKIP\_JETTIES}
\\Bei gesetztem Wert `1` werden keine Buhnen importiert.

\textbf{SKIP\_FLOODMARKS}
\\Bei gesetztem Wert `1` werden keine HW-Marken importiert.

\subsection{Starten des Geodaten Importers}
\label{Starten des Geodaten Importers}
Der Geodaten Importer wird mittels des Shellskripts,
dass auch für die Konfiguration verwendet wird, von einer Konsole
gestartet. Dazu führen Sie folgenden Befehl aus:\\

\begin{lstlisting}
    sh ./run_geo.sh pfad/zur/beispiel.gew > geo-import.log
\end{lstlisting}

Bezüglich des übergebenen Pfades siehe auch Kapitel \ref{start-hydr}.
Der Importer wird nun gestartet. Sämtliche Log-Ausgaben werden in die Datei
$geo-import.log$ geschrieben.


